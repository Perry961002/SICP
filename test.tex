\documentclass[]{article}
\usepackage{lmodern}
\usepackage{amssymb,amsmath}
\usepackage{ifxetex,ifluatex}
\usepackage{fixltx2e} % provides \textsubscript
\ifnum 0\ifxetex 1\fi\ifluatex 1\fi=0 % if pdftex
  \usepackage[T1]{fontenc}
  \usepackage[utf8]{inputenc}
\else % if luatex or xelatex
  \ifxetex
    \usepackage{mathspec}
  \else
    \usepackage{fontspec}
  \fi
  \defaultfontfeatures{Ligatures=TeX,Scale=MatchLowercase}
\fi
% use upquote if available, for straight quotes in verbatim environments
\IfFileExists{upquote.sty}{\usepackage{upquote}}{}
% use microtype if available
\IfFileExists{microtype.sty}{%
\usepackage[]{microtype}
\UseMicrotypeSet[protrusion]{basicmath} % disable protrusion for tt fonts
}{}
\PassOptionsToPackage{hyphens}{url} % url is loaded by hyperref
\usepackage[unicode=true]{hyperref}
\hypersetup{
            pdfborder={0 0 0},
            breaklinks=true}
\urlstyle{same}  % don't use monospace font for urls
\usepackage{color}
\usepackage{fancyvrb}
\newcommand{\VerbBar}{|}
\newcommand{\VERB}{\Verb[commandchars=\\\{\}]}
\DefineVerbatimEnvironment{Highlighting}{Verbatim}{commandchars=\\\{\}}
% Add ',fontsize=\small' for more characters per line
\newenvironment{Shaded}{}{}
\newcommand{\KeywordTok}[1]{\textcolor[rgb]{0.00,0.44,0.13}{\textbf{#1}}}
\newcommand{\DataTypeTok}[1]{\textcolor[rgb]{0.56,0.13,0.00}{#1}}
\newcommand{\DecValTok}[1]{\textcolor[rgb]{0.25,0.63,0.44}{#1}}
\newcommand{\BaseNTok}[1]{\textcolor[rgb]{0.25,0.63,0.44}{#1}}
\newcommand{\FloatTok}[1]{\textcolor[rgb]{0.25,0.63,0.44}{#1}}
\newcommand{\ConstantTok}[1]{\textcolor[rgb]{0.53,0.00,0.00}{#1}}
\newcommand{\CharTok}[1]{\textcolor[rgb]{0.25,0.44,0.63}{#1}}
\newcommand{\SpecialCharTok}[1]{\textcolor[rgb]{0.25,0.44,0.63}{#1}}
\newcommand{\StringTok}[1]{\textcolor[rgb]{0.25,0.44,0.63}{#1}}
\newcommand{\VerbatimStringTok}[1]{\textcolor[rgb]{0.25,0.44,0.63}{#1}}
\newcommand{\SpecialStringTok}[1]{\textcolor[rgb]{0.73,0.40,0.53}{#1}}
\newcommand{\ImportTok}[1]{#1}
\newcommand{\CommentTok}[1]{\textcolor[rgb]{0.38,0.63,0.69}{\textit{#1}}}
\newcommand{\DocumentationTok}[1]{\textcolor[rgb]{0.73,0.13,0.13}{\textit{#1}}}
\newcommand{\AnnotationTok}[1]{\textcolor[rgb]{0.38,0.63,0.69}{\textbf{\textit{#1}}}}
\newcommand{\CommentVarTok}[1]{\textcolor[rgb]{0.38,0.63,0.69}{\textbf{\textit{#1}}}}
\newcommand{\OtherTok}[1]{\textcolor[rgb]{0.00,0.44,0.13}{#1}}
\newcommand{\FunctionTok}[1]{\textcolor[rgb]{0.02,0.16,0.49}{#1}}
\newcommand{\VariableTok}[1]{\textcolor[rgb]{0.10,0.09,0.49}{#1}}
\newcommand{\ControlFlowTok}[1]{\textcolor[rgb]{0.00,0.44,0.13}{\textbf{#1}}}
\newcommand{\OperatorTok}[1]{\textcolor[rgb]{0.40,0.40,0.40}{#1}}
\newcommand{\BuiltInTok}[1]{#1}
\newcommand{\ExtensionTok}[1]{#1}
\newcommand{\PreprocessorTok}[1]{\textcolor[rgb]{0.74,0.48,0.00}{#1}}
\newcommand{\AttributeTok}[1]{\textcolor[rgb]{0.49,0.56,0.16}{#1}}
\newcommand{\RegionMarkerTok}[1]{#1}
\newcommand{\InformationTok}[1]{\textcolor[rgb]{0.38,0.63,0.69}{\textbf{\textit{#1}}}}
\newcommand{\WarningTok}[1]{\textcolor[rgb]{0.38,0.63,0.69}{\textbf{\textit{#1}}}}
\newcommand{\AlertTok}[1]{\textcolor[rgb]{1.00,0.00,0.00}{\textbf{#1}}}
\newcommand{\ErrorTok}[1]{\textcolor[rgb]{1.00,0.00,0.00}{\textbf{#1}}}
\newcommand{\NormalTok}[1]{#1}
\IfFileExists{parskip.sty}{%
\usepackage{parskip}
}{% else
\setlength{\parindent}{0pt}
\setlength{\parskip}{6pt plus 2pt minus 1pt}
}
\setlength{\emergencystretch}{3em}  % prevent overfull lines
\providecommand{\tightlist}{%
  \setlength{\itemsep}{0pt}\setlength{\parskip}{0pt}}
\setcounter{secnumdepth}{0}
% Redefines (sub)paragraphs to behave more like sections
\ifx\paragraph\undefined\else
\let\oldparagraph\paragraph
\renewcommand{\paragraph}[1]{\oldparagraph{#1}\mbox{}}
\fi
\ifx\subparagraph\undefined\else
\let\oldsubparagraph\subparagraph
\renewcommand{\subparagraph}[1]{\oldsubparagraph{#1}\mbox{}}
\fi

% set default figure placement to htbp
\makeatletter
\def\fps@figure{htbp}
\makeatother


\date{}

\begin{document}

\begin{Shaded}
\begin{Highlighting}[]
\CommentTok{;将两个序对交替取出}
\CommentTok{; s1,t1,s2,t2,...,si,ti,...}
\NormalTok{(}\KeywordTok{define}\FunctionTok{ }\NormalTok{(interleave s t)}
\NormalTok{    (}\KeywordTok{if}\NormalTok{ (stream-empty? s)}
\NormalTok{        t}
\NormalTok{        (stream-cons (stream-car s)}
\NormalTok{                     (interleave t (stream-cdr s)))))}
\CommentTok{;序对流}
\NormalTok{(}\KeywordTok{define}\FunctionTok{ }\NormalTok{(pairs s t)}
\NormalTok{    (stream-cons}
\NormalTok{        (}\KeywordTok{list}\NormalTok{ (stream-car s) (stream-car t))}
\NormalTok{        (interleave}
\NormalTok{            (stream-map (}\KeywordTok{lambda}\NormalTok{ (x) (}\KeywordTok{list}\NormalTok{ (stream-car s) x))}
\NormalTok{                        (stream-cdr t))}
\NormalTok{            (pairs (stream-cdr s) (stream-cdr t)))))}

\CommentTok{;(display-top10 (pairs integers integers))}
\CommentTok{;(1 1)  (1 2)  (2 2)  (1 3)  (2 3)  (1 4)  (3 3)  (1 5)  (2 4) }
\end{Highlighting}
\end{Shaded}

\begin{itemize}
\item
  以\(Pa(S, T)\)表示序对\((S_i, T_j)\)组成的流,其中\((i \le j\  且\ i, j > 0)\)。对这个流有如下描述:

  流的第一个序对是
  \((S_1, T_1)\),其他部分是流\( \begin{Bmatrix}(S_1, T_2), (S_1, T_3), \cdots\end{Bmatrix} \)和
  \(Pa(S.r, T.r) \)中的序对元素按交叉顺序出现的流。(其中\(S.r\)表示流\(S\)中第一个元素之外的其他元素,用
  \(\langle a, b \rangle\) 表示流 \(a,\ b\)
  中的元素按交叉顺序出现),则有

  \[Pa(S, T) = \begin{Bmatrix}(S_1, T_1), \langle\ \begin{Bmatrix}(S_1, T_2), (S_1, T_3), \cdots\end{Bmatrix},Pa(S.r, T.r)\  \rangle \end{Bmatrix}\]
\end{itemize}

\begin{itemize}
\item
  用 \(F(i, j)\) 表示序对\( (S_i, T_j)\) 到序对\( (S_1, T_1)\)
  的距离,结合上面的描述,我们递归的给出这个函数表达式:

   假设分别从 \(S \)和 \(T\) 的第 \(i-1 \)和
  \(j-1 \)的位置开始构造,可以得到下面的一个流(为了方便,这里仅用下标表示元素):

  \[\cdots, (i-1, j-1), (i-1, j), (i, j), (i-1, j+1), \cdots\]

   我们知道序对\( (i-1, j-1) \)到首序对的距离为
  \(F(i-1, j-1)\),结合\(Pa(S, T)\)的数学描述,明显的我们可以得到

  \[F(i ,j) = 2 * F(i-1, j-1) + 2\]

  特殊的,我们知道 \(F(1, 1) = 0,F(1, j) = 2 * j - 3,(j > 1)\)。

  一般的,当 \( i = j\) 时,有

  \begin{eqnarray*}
  F(i ,j) &=& 2 * F(i-1, j-1) + 2\\
  &\vdots&\\
  \         &=& 2^{(i-1)}*F(1, 1) + 2 + 2^2 + ... + 2^{(i-1)}\\
  &=& 2^i - 2\\
  \end{eqnarray*}
\end{itemize}

 当 \( i < j\) 时,我们可以知道 \( i \) 一定会先被减到 \(1\)
,所以我们可以得到

\begin{eqnarray*}
  F(i ,j) &=& 2^{(i-1)}*F(1, j-i+1) + 2 + 2^2 + ... + 2^{(i-1)}\\
  &\vdots&\\
  &=& 2^i * (j-i) + 2^{(i-1)}-2
  \end{eqnarray*}

\begin{itemize}
\item
  按照上面的\(F(1, 100) = 197\),并且代码验证
  \texttt{(stream-ref\ (pairs\ integers\ integers)\ 197)},得到\texttt{(1,\ 100)}
\item
  那么\(F(100, 100) = 2^{100} - 2\)
\end{itemize}

\end{document}
